\section{Introduction}

Mother Nature has created a diverse set of awe-inspiring motions in the animal kingdom: Birds can fly in the sky, fishes can swim in the water, geckos can crawl on vertical surfaces, and cats can reorient themselves in mid-air. Similarly, human's motions exhibit efficiency (locomotion), agility (Kung Fu), gracefulness (Ballet) and dexterity (hand manipulation). Studying these motions is not only a scientific quest that quenches our curiosity, but also an important step towards synthesizing them in a way that can fundamentally change our life. Character animation aims to faithfully synthesize these motions of humans and animals, and display them to an audience for the purpose of entertainment, story-telling and education. The synthesized motions need to appear realistic to give the audience an immersive experience. 

In the last few decades, we have seen tremendous advance in character animation. Some of the most breathtaking movies, such as Harry Potter, Avatar and Life of Pi, rely heavily on computer generated animations. Nowadays, it is almost impossible for the audience to tell apart the computer-synthesized motions from the real footage. Behind these realistic animations lies countless hours of tedious manual work of highly-specialized experts. For example, to produce a 100-minute feature film at Pixar can take dozens of artists and engineers more than five years of development. In today's animation pipeline, the most popular techniques are key frames or motion capture, both of which require artistic expertise and laborious manual work. Even worse, the knowledge and efforts that are put into one animation sequence are not necessarily generalizable to other motions. In my point of view, these are not efficient or principled ways of animation synthesis.

\textbf{A principled way to synthesize character animation is to study the fundamental factors that have shaped our motions.} Instead of focusing on the appearance of our motions, we need to dig deeper to understand why we move in the way that we are doing today. After understanding the root causes that have shaped our movements, we can then synthesize them naturally and automatically. Our motions are shaped through millions of years of optimization (evolution) in a world that obey physical laws. This insight has motivated a new paradigm of \emph{Physically-Based Character Animation}. The two key components of this paradigm are physical simulation\index{simulation} and motion control\index{control}. We first build a physical simulation to model the physical world and then perform optimization to control the motions of characters so that they can move purposefully, naturally and robustly in the simulated environment. 

Although we often take our motions for granted since we can perform them so effortlessly, physically-based character animation is a notoriously difficult problem because our motions involve sophisticated neuromuscular control, sensory information processing, motion planning, coordinated muscle activation, and complicated interactions with the surrounding physical environment. Even though we are still far from fully understand the underlying control mechanisms that govern our motions, two decades of research in physically-based character animation has brought us new insights, effective methodologies and impressive results. The purpose of this chapter is to review the state of the art (Section 2), introduce some of the established algorithms (Section 3 and 4), and discuss promising future research directions (Section 5) in physically-based character animation.

